% Mini Project Report - Kuma: AI-Powered Personal Assistant
% Main Report File

\documentclass[12pt,a4paper,twoside]{report}

% Package imports
\usepackage{graphicx}
\usepackage{amsmath}
\usepackage{amsfonts}
\usepackage{color}
\usepackage{amssymb}
\usepackage{chngcntr}
\usepackage{cite}
\usepackage{fancyhdr}
\usepackage[export]{adjustbox}
\usepackage[compact]{titlesec}
\usepackage{tabularx}
\usepackage{layout}
\usepackage{enumerate}
\usepackage{enumitem}
\usepackage{mathtools}
\usepackage[toc,page]{appendix}
\usepackage[normalem]{ulem}
\usepackage{booktabs}
\usepackage{rotating}
\usepackage{longtable}
\usepackage{caption}
\usepackage{pdflscape}
\usepackage[english]{babel}
\usepackage[autostyle, english = american]{csquotes}
\usepackage{listings}
\usepackage{xcolor}

\MakeOuterQuote{"}

% Page layout settings
\setlength{\voffset}{-0.75in}
\setlength{\headsep}{10pt}
\setlength{\parindent}{0cm}

% Title formatting
\titlespacing{\section}{0pt}{0pt}{0pt}
\counterwithin{figure}{chapter}
\numberwithin{equation}{chapter}
\renewcommand{\baselinestretch}{1.2}
\renewcommand{\thetable}{\arabic{chapter}.\arabic{table}} 
\renewcommand{\footrulewidth}{0.4pt}
\textheight 245mm \textwidth 160mm \topmargin 15mm
\setlength{\oddsidemargin}{.1in}
\evensidemargin .1in 

\titleformat{\chapter}[display]
{\normalfont\huge\bfseries}{\chaptertitlename\ \thechapter}{0pt}{\Huge}
\titlespacing*{\chapter}{0pt}{0pt}{0pt}

% Header and footer settings
\pagestyle{fancy}
\cfoot{}
\lhead{{\footnotesize Kuma: AI-Powered Personal Assistant}}
\lfoot{Dept.of CSE, S.I.T.,Tumakuru-03}
\rfoot{\thepage}
\rhead{2025-26}
\linespread{1.5}
\numberwithin{equation}{chapter}
\let\cleardoublepage\clearpage
\bibliographystyle{unsrt}

% Code listing settings
\lstset{
    basicstyle=\ttfamily\small,
    breaklines=true,
    frame=single,
    numbers=left,
    numberstyle=\tiny,
    captionpos=b
}

\begin{document}
\addtocontents{toc}{\protect\thispagestyle{empty}}

% ==================== COVER PAGE ====================
\begin{titlepage}
\thispagestyle{empty}
\centering
{\textbf{SIDDAGANGA INSTITUTE OF TECHNOLOGY, TUMAKURU-572103}} \\
\textbf{{\small (An Autonomous Institute under Visvesvaraya Technological University, Belagavi)}}
$$\includegraphics[scale=1.0]{Logo}$$
\begin{center}
{\Large \textbf{Project Report on}}\\[0.5cm]

{\color{black}{{\textbf{\Large\color{red}"Kuma: AI-Powered Personal Assistant"}}}} \\[0.5cm]
{{\large submitted in partial fulfillment of the requirement for the completion of V semester of}} \\
{\textbf{\large BACHELOR OF ENGINEERING\\ in\\ ARTIFICIAL INTELLIGENCE AND DATA SCIENCE\\}} 
{\textbf{\large Submitted by}}\\[0.7cm]
\end{center}

\begin{center}
\begin{tabular}{ll}
\color{blue} Suraj Kumar & \color{blue} (1SI23AD057)  \\
\color{blue} Himanshu Rai & \color{blue} (1SI23AD016)  \\
\color{blue} Aditya Raj & \color{blue} (1SI23CS008)  \\
\end{tabular}\\[1.0cm]
\end{center}

\begin{center}
under the guidance of

\textbf{\color{green} Dr Sheela S}\\
Assistant Professor\\
Department of Computer Science and Engineering\\
SIT, Tumakuru-03\\[1.0cm]
\end{center}

{\textbf{{{\small DEPARTMENT OF COMPUTER SCIENCE AND ENGINEERING \\2025-26}}}}
\end{titlepage}

% ==================== CERTIFICATE ====================
\begin{titlepage}
\begin{center}
{\textbf{SIDDAGANGA INSTITUTE OF TECHNOLOGY, TUMAKURU-572103}} \\
{(An Autonomous Institute under Visvesvaraya Technological University, Belagavi)}
\textbf{{\small{DEPARTMENT OF COMPUTER SCIENCE AND ENGINEERING}}}
$$\includegraphics[scale=1.0]{Logo}$$

{\color{red}{\Large \bf CERTIFICATE}}\\[0.5cm]
\end{center}

Certified that the mini project work entitled {\color{blue}"KUMA: AI-POWERED PERSONAL ASSISTANT"} is a bonafide work carried out by Suraj Kumar (1SI23AD057), Himanshu Rai (1SI23AD016) and Aditya Raj (1SI23CS008) in partial fulfillment for the completion of V Semester of Bachelor of Engineering in Artificial Intelligence and Data Science from Siddaganga Institute of Technology, an autonomous institute under Visvesvaraya Technological University, Belagavi during the academic year 2025-26. It is certified that all corrections/suggestions indicated for internal assessment have been incorporated in the report deposited in the department library. The Mini project report has been approved as it satisfies the academic requirements in respect of project work prescribed for the Bachelor of Engineering degree.\\[1.0cm] 

\begin{tabular}{lll}
Dr Sheela S    \hspace{8cm} & Head of the Department \hspace{8cm}        \\
Assistant Professor      & Dept. of CSE          \\
Dept. of CSE   & SIT,Tumakuru-03        &                 \\
SIT,Tumakuru-03  &                        &                
\end{tabular}\\[1.0cm]

\textbf{External viva:}\\
\textbf{Names of the Examiners} \hspace{6.0cm} \textbf{Signature with date}\\
\textbf{1.}\\
\textbf{2.}\\
\end{titlepage}

% ==================== ACKNOWLEDGEMENT ====================
\begin{titlepage}
\begin{center}
{\Large \textbf{ACKNOWLEDGEMENT}}
\end{center}

We offer our humble pranams at the lotus feet of \textbf{His} Holiness, \textbf{Dr. Sree Sree Sivakumara Swamigalu}, Founder President and \textbf{His} Holiness, \textbf{Sree Sree Siddalinga Swamigalu}, President, Sree Siddaganga Education Society, Sree Siddaganga Math for bestowing upon their blessings.\\

We deem it as a privilege to thank \textbf{Dr. Shivakumaraiah}, CEO, SIT, Tumakuru, and \textbf{Dr. S V Dinesh}, Principal, SIT, Tumakuru for fostering an excellent academic environment in this institution, which made this endeavor fruitful.\\

We would like to express our sincere gratitude to \textbf{Dr Sunitha. N R}, Professor and Head, Department of Computer Science and Engineering, SIT, Tumakuru for her encouragement and valuable suggestions.\\

We thank our guide \textbf{Dr Sheela S}, Assistant Professor, Department of Computer Science and Engineering, SIT, Tumakuru for the valuable guidance, advice and encouragement.\\

\begin{flushright}
\begin{tabular}{ll}
Suraj Kumar & (1SI23AD057)  \\
Himanshu Rai & (1SI23AD016)  \\
Aditya Raj & (1SI23CS008)  \\
\end{tabular}\\[1.5cm]
\end{flushright}
\end{titlepage}

% ==================== COURSE OUTCOMES ====================
\begin{titlepage}
\begin{center}
{\Large \textbf{Course Outcomes}}
\end{center}

CO1: To identify a problem through literature survey and knowledge of contemporary engineering technology. \\      
	% PO12: Life  long learning (3)\\
CO2: To consolidate the literature search to identify issues/gaps and formulate the engineering problem \\
%PO2  Problem analysis(3)\\
CO3: To prepare project schedule for the identified design methodology and engage in budget analysis, and share responsibility for every member in the team\\
%PO11  Project management and finance (3)\\
CO4: To provide sustainable  engineering solution considering health, safety, legal, cultural issues and also demonstrate concern for environment \\
%PO6 Engineer and Society (3)\\
%PO7 Environment and sustainability (3)\\
CO5: To identify and apply the mathematical concepts, science concepts, engineering and management concepts necessary to implement the identified engineering problem \\
%PO1 Engineering Knowledge \\
%PO2 Problem Analysis(3)
CO6: To select the engineering tools/components required to implement the proposed solution for the identified engineering problem\\
%PO5 Modern tool Usage(3) \\
CO7: To analyze, design, and implement optimal design solution, interpret results of  experiments and draw valid conclusion \\
%PO3 Design/Development of Solutions(3) \\
%PO4 Conduct investigation of complex problems(3)\\
CO8: To demonstrate effective written communication through the project report, the one-page poster presentation, and preparation of the video about the project and the four page IEEE/Springer/ paper format of the work\\
%PO10 Communication (3)\\
CO9: To engage in effective oral communication through power point presentation and demonstration of the project work \\
%PO10 Communication(3)\\
CO10:To demonstrate compliance to the prescribed standards/ safety norms and abide by the norms of professional ethics\\
%PO8 Ethics(3)\\
CO11: To perform in the team, contribute to the team and mentor/lead the team
%PO9 Individual and Team work (3)\\



%CO 1 : Identify , formulate the problem and define the objectives\\
%CO 2 : Review the literature and provide efficient design solution with appropriate consideration for societal, health and safety issues\\
%CO 3 : Select the engineering tools/components and develop an  experimental setup to validate the design \\
%CO 4 : Test, analyse and interpret the results of the experiments in compliance with the  defined objectives\\
%CO 5 : Document as per the standard, present effectively the work following professional ethics and interact with target group\\
%CO 6 : Contribute to the team, lead the diverse team, demonstrating engineering and management principles\\
%CO 7 : Demonstrate engineering and management principles, perform the budget analysis through utilization of the resources (finance, power, area, bandwidth, weight, size, etc)\\
\begin{table}[h!]
\caption*{\textbf{CO-PO Mapping}}
\resizebox{\textwidth}{!}{%
\begin{tabular}{|l|l|l|l|l|l|l|l|l|l|l|l|l|l|l|}
\hline
 & PO1 & PO2 & PO3 & PO4 & PO5 & PO6 & PO7 & PO8 & PO9 & PO10 & PO11 & PSO1 & PSO2 & PSo3 \\ \hline
CO-1 &  &  &  &  &  &  &  &  &  &  &   3 &  & 3  &\\ \hline
CO-2 &  &3  &  &3  &  &  &  &  &  &  &    &  &  3 &\\ \hline
CO-3 &  &  &  &  &  &  &  &  &  & 3 & 3   &  & 3  & \\ \hline
CO-4 &  &  &  &  &  & 3 & 3 &  &  &    &  &  & 3  & \\ \hline
CO-5 & 3 & 3 &  &  &  &  &  &  &  & &  &  & 3  &\\ \hline
CO-6 &  &  &  &  & 3 &  &  &  &  &  &    &  & 3  &\\ \hline
CO-7 &  & 3 & 3 & 3 &  &  &  &  &  &  &    & & 3  &\\ \hline
CO-8 &  &  &  &  &  &  &  &  & 3 &  &    &  &3  &\\ \hline
CO-9 &  &  &  &  &  &  &  &  & 3 &  &    &  &  3&\\ \hline
CO-10 &  &  &  &  &  &  &3  &  &  &  &    &  &  3 &\\ \hline
CO-11 &  &  &  &  &  &  &  & 3 &  &  &   &  & 3  & \\ \hline

\end{tabular}%
}
\end{table}

\vspace{10px}
\\Attainment level: - 1: Slight (low) 2: Moderate (medium)	3: Substantial (high)
\\POs: PO1: Engineering Knowledge, PO2: Problem analysis, PO3: Design/Development of solutions, PO4: Conduct investigations of complex problems, PO5: Engineering tool usage, PO6: Engineer and the world, PO7: Ethics, PO8: Individual and collaborative team work, PO9: Communication, PO10: Project management and finance, PO11: Lifelong learning\\PSO1:Computer based systems development,PSO2:Software development,PSO3:Computer Communications and Internet applications
\end{titlepage}

% ==================== ABSTRACT ====================
\chapter*{Abstract}
\thispagestyle{plain}
\addcontentsline{toc}{chapter}{\numberline{}Abstract}

[Add abstract here covering three paragraphs:

\textbf{Paragraph 1 - Motivation:} Discuss the growing need for intelligent personal assistants that can handle diverse tasks across productivity tools, voice interaction, and document analysis. Highlight limitations of existing solutions and the opportunity for a customizable, self-hosted alternative.

\textbf{Paragraph 2 - Objectives:} State the main goals including development of a multi-agent AI system, voice-enabled interaction with Indic language support, vision capabilities for image and document analysis, scalable architecture using Redis message queues, and containerized deployment using Docker.

\textbf{Paragraph 3 - Implementation:} Mention key technologies: TypeScript and Bun runtime for backend, React and Vite for frontend, Google Gemini and OpenAI for AI processing, Sarvam AI for voice, Redis Streams for message queuing, PostgreSQL with Prisma ORM for data persistence, and Docker for containerized deployment. Highlight the multi-agent architecture with specialized agents for different task domains.]

% ==================== TABLE OF CONTENTS ====================
\pagenumbering{roman}
\tableofcontents
\thispagestyle{empty}

% ==================== LIST OF FIGURES ====================
\listoffigures
\addcontentsline{toc}{chapter}{\numberline{}List of Figures}

% ==================== LIST OF TABLES ====================
\listoftables
\addcontentsline{toc}{chapter}{\numberline{}List of Tables}

\cleardoublepage

% ==================== MAIN CONTENT ====================
\pagenumbering{arabic}
\setcounter{page}{1}

% ==================== CHAPTER 1: INTRODUCTION ====================
\chapter{Introduction}

\section{Motivation}
[Discuss the following points:
\begin{itemize}
    \item Growing complexity of digital workflows requiring intelligent automation
    \item Fragmentation of productivity tools (email, calendar, documents) needing unified access
    \item Limitations of existing assistants: closed ecosystems, limited customization, privacy concerns
    \item Emergence of powerful LLMs enabling sophisticated conversational agents
    \item Need for multimodal interaction combining text, voice, and visual inputs
    \item Importance of self-hosted solutions for data privacy and customization
    \item Opportunity to leverage open-source AI frameworks for building personalized assistants
    \item Growing demand for Indic language support in voice-based interfaces
\end{itemize}]

\section{Objective of the project}
[State the primary objectives of the Kuma project:
\begin{itemize}
    \item Development of an intelligent personal assistant using modern AI technologies
    \item Implementation of multi-agent architecture with specialized agents for different tasks
    \item Voice-enabled interaction with speech-to-text and text-to-speech capabilities
    \item Image understanding and vision-based document analysis
    \item Integration of productivity services (Gmail, Calendar, Docs, Drive, GitHub)
    \item Scalable message queue architecture for reliable AI processing
    \item Containerized deployment using Docker for production readiness
    \item Creating a responsive and intuitive user interface
\end{itemize}]

\section{Organisation of the report}
[Describe the structure of the report:
\begin{itemize}
    \item Chapter 1: Introduction covering motivation, objectives, and report organization
    \item Chapter 2: Literature survey on conversational AI, agent frameworks, voice processing, and containerization
    \item Chapter 3: System design including architecture, data flow, Redis queue system, and Docker deployment
    \item Chapter 4: Implementation covering backend services, AI agents, voice pipeline, vision processing, and frontend
    \item Chapter 5: Results with screenshots, performance benchmarks, and comparative analysis
    \item Chapter 6: Conclusion summarizing achievements and outlining future enhancements
    \item Bibliography and Appendices with SDG mapping and component data sheets
\end{itemize}]

% ==================== CHAPTER 2: LITERATURE SURVEY ====================
\chapter{Literature Survey}

\section{Conversational AI and Virtual Assistants}
[Review existing AI assistants - Google Assistant, Siri, Alexa, ChatGPT. Discuss their capabilities, limitations, and architectural approaches. Analyze trends in conversational AI research.]

\section{Large Language Models and Agent Frameworks}
[Explore LLM architectures and their evolution. Discuss agent frameworks including LangChain, AutoGPT, and ReAct pattern. Review multi-agent coordination strategies.]

\section{Google Gemini and Multimodal AI}
[Analyze Google Gemini's multimodal capabilities for text, image, and audio processing. Compare with GPT-4V and other vision-language models.]

\section{Speech Processing Technologies}
[Review speech-to-text and text-to-speech technologies. Discuss real-time voice communication protocols and WebRTC/LiveKit frameworks. Analyze Indic language support in voice systems.]

\section{Image Understanding and Vision AI}
[Explore computer vision for document analysis. Review OCR technologies and vision-language models for image captioning and visual question answering.]

\section{Message Queue Architectures}
[Discuss asynchronous processing patterns using Redis Streams, RabbitMQ, and Kafka. Analyze producer-consumer models and dead letter queue strategies.]

\section{Containerization and Deployment}
[Review Docker containerization benefits and multi-stage build patterns. Discuss container orchestration and microservices deployment strategies.]

\section{Web Application Technologies}
[Analyze modern web stack choices - React, TypeScript, Vite for frontend. Review Bun runtime, Express, and Prisma ORM for backend development. Mention JWT for user sessions and OAuth 2.0 for third-party service integration.]

[Include citations from IEEE, Springer, ACM, and other peer-reviewed journals. Use proper citation format: \cite{langchain}, \cite{gemini}.]

% ==================== CHAPTER 3: SYSTEM DESIGN & METHODOLOGY ====================
\chapter{System Design \& Methodology}

\section{Functional \& Non-Functional Requirements}

\subsection{Functional Requirements}
[List functional requirements:
\begin{enumerate}
    \item User login and session management
    \item Real-time chat interface with streaming AI responses
    \item Multi-agent system with intelligent routing to specialized agents
    \item Voice input processing with speech-to-text transcription
    \item Voice output generation with text-to-speech synthesis
    \item Image upload and vision-based analysis
    \item Document upload with PDF text extraction and summarization
    \item Integration with Google services (Gmail, Calendar, Docs, Drive)
    \item GitHub repository interaction capabilities
    \item Long-term memory storage and contextual retrieval
    \item Web search and information gathering tools
    \item Asynchronous message processing via Redis queue
    \item Docker-based containerized deployment
\end{enumerate}]

\subsection{Non-Functional Requirements}
[List non-functional requirements:
\begin{enumerate}
    \item Performance: AI response generation within 3 seconds, voice latency under 500ms
    \item Security: Basic user authentication and secure API access
    \item Scalability: Horizontal scaling via Redis queue and containerized deployment
    \item Reliability: Health checks, automatic container restarts, dead letter queue for failed jobs
    \item Usability: Responsive interface supporting both text and voice input
    \item Maintainability: Modular architecture with separate services for API, worker, and frontend
    \item Portability: Docker containers enabling consistent deployment across environments
\end{enumerate}]

\section{List of Hardware \& Software Requirements}

\subsection{Hardware Requirements}
[Specify hardware requirements:
\begin{itemize}
    \item Processor: Intel Core i5 or equivalent (minimum)
    \item RAM: 8 GB (minimum), 16 GB (recommended)
    \item Storage: 20 GB free space
    \item Network: Stable internet connection
\end{itemize}]

\subsection{Software Requirements}
[Specify software requirements:
\begin{itemize}
    \item Operating System: Windows 10/11, macOS, or Linux
    \item Runtime: Bun >= 1.0.0
    \item Database: PostgreSQL
    \item Node.js: Version 18+ (optional)
    \item Browser: Chrome, Firefox, Safari (latest versions)
    \item Development Tools: VS Code or similar IDE
    \item Version Control: Git
\end{itemize}

\textbf{Key Technologies:}
\begin{itemize}
    \item Backend: TypeScript, Bun runtime, Express framework, Prisma ORM
    \item Frontend: React 18, Vite, TypeScript, Zustand state management, shadcn/ui components
    \item AI/ML: Vercel AI SDK, Google Gemini API, OpenAI API, Supermemory for long-term memory
    \item Voice: Sarvam AI for Indic speech-to-text and text-to-speech, LiveKit for real-time communication
    \item Vision: Google Gemini Vision for image analysis and document understanding
    \item Queue: Redis Streams for asynchronous message processing
    \item Containerization: Docker, Docker Compose, NGINX for production serving
    \item APIs: Google OAuth 2.0, Gmail API, Calendar API, Drive API, Docs API, GitHub API
\end{itemize}]

\section{System Architecture}
[Describe the overall system architecture. Include:
\begin{itemize}
    \item Three-tier architecture with React frontend, Express API, and PostgreSQL database
    \item Client-Server communication via REST API and Server-Sent Events for streaming
    \item Redis-based message queue for asynchronous AI processing
    \item Multi-agent system with router agent delegating to specialized agents
    \item Voice pipeline connecting Sarvam STT/TTS with AI agents
    \item Vision module for image analysis and document processing
    \item OAuth integration layer for Google and GitHub services
    \item Docker containerization with separate containers for API, worker, and frontend
\end{itemize}

Refer to Figure \ref{fig:architecture} for the system architecture diagram.]

\begin{figure}[h]
\centering
% \includegraphics[width=\textwidth]{architecture_diagram}
\caption{System Architecture of Kuma AI Assistant}
\label{fig:architecture}
\end{figure}

\section{Redis Queue Architecture}
[Describe the asynchronous message processing system:
\begin{itemize}
    \item Producer-consumer pattern using Redis Streams
    \item Message job lifecycle: pending, processing, completed, failed
    \item Consumer groups for reliable message delivery
    \item Dead letter queue for failed message handling
    \item Status tracking and real-time updates via pub/sub
\end{itemize}

Refer to Figure \ref{fig:redis} for the Redis queue flow diagram.]

\begin{figure}[h]
\centering
% \includegraphics[width=\textwidth]{redis_architecture}
\caption{Redis Message Queue Architecture}
\label{fig:redis}
\end{figure}

\section{Voice Processing Architecture}
[Describe the voice interaction pipeline:
\begin{itemize}
    \item Audio capture and streaming using LiveKit
    \item Speech-to-text conversion using Sarvam AI
    \item AI agent processing of transcribed text
    \item Text-to-speech synthesis for voice responses
    \item Real-time bidirectional communication flow
\end{itemize}

Refer to Figure \ref{fig:voice} for the voice processing flow diagram.]

\begin{figure}[h]
\centering
% \includegraphics[width=\textwidth]{voice_architecture}
\caption{Voice Processing Pipeline}
\label{fig:voice}
\end{figure}

\section{Docker Deployment Architecture}
[Describe the containerized deployment setup:
\begin{itemize}
    \item Multi-stage Dockerfile builds for optimized images
    \item Docker Compose orchestration of all services
    \item Service containers: frontend (NGINX), backend (Bun), worker, PostgreSQL, Redis
    \item Volume mounts for persistent data storage
    \item Health checks and automatic restart policies
    \item Network isolation using Docker bridge networks
\end{itemize}

Refer to Figure \ref{fig:docker} for the Docker deployment diagram.]

\begin{figure}[h]
\centering
% \includegraphics[width=\textwidth]{docker_architecture}
\caption{Docker Deployment Architecture}
\label{fig:docker}
\end{figure}

\section{Data Flow Diagrams}

\subsection{Level 0 DFD - Context Diagram}
[Describe the context-level data flow showing the system as a single process with external entities:
\begin{itemize}
    \item User (text, voice, image inputs)
    \item Google Services (Gmail, Calendar, Docs, Drive)
    \item AI APIs (Gemini, OpenAI, Supermemory)
    \item Voice Services (Sarvam STT/TTS, LiveKit)
\end{itemize}]

\begin{figure}[h]
\centering
% \includegraphics[width=\textwidth]{dfd_level0}
\caption{Context Diagram (Level 0 DFD)}
\label{fig:dfd0}
\end{figure}

\subsection{Level 1 DFD}
[Describe the Level 1 DFD showing major processes:
\begin{itemize}
    \item Authentication and Session Management
    \item Chat and Message Processing
    \item AI Agent Routing and Execution
    \item Voice Input/Output Processing
    \item Image and Document Analysis
    \item External Service Integration
    \item Redis Queue Management
\end{itemize}]

\begin{figure}[h]
\centering
% \includegraphics[width=\textwidth]{dfd_level1}
\caption{Level 1 Data Flow Diagram}
\label{fig:dfd1}
\end{figure}

\subsection{Level 2 DFD - Agent Processing}
[Describe detailed data flow within the agent processing module:
\begin{itemize}
    \item Router agent receiving user query
    \item Agent selection based on query classification
    \item Tool invocation and external API calls
    \item Response generation and formatting
    \item Memory storage and context retrieval
\end{itemize}]

\begin{figure}[h]
\centering
% \includegraphics[width=\textwidth]{dfd_level2}
\caption{Level 2 DFD - Agent Processing Module}
\label{fig:dfd2}
\end{figure}

\subsection{Level 2 DFD - Voice Processing}
[Describe detailed data flow for voice interactions:
\begin{itemize}
    \item Audio stream capture and buffering
    \item Speech recognition and transcription
    \item Text processing by AI agent
    \item Speech synthesis and audio streaming
\end{itemize}]

\begin{figure}[h]
\centering
% \includegraphics[width=\textwidth]{dfd_voice}
\caption{Level 2 DFD - Voice Processing Module}
\label{fig:dfd_voice}
\end{figure}

\subsection{Level 2 DFD - Image Processing}
[Describe detailed data flow for vision capabilities:
\begin{itemize}
    \item Image upload and validation
    \item Vision model analysis (Gemini Vision)
    \item Text extraction (OCR) from documents
    \item Scene description and object detection
    \item Response generation with visual context
\end{itemize}]

\begin{figure}[h]
\centering
% \includegraphics[width=\textwidth]{dfd_vision}
\caption{Level 2 DFD - Image Processing Module}
\label{fig:dfd_vision}
\end{figure}

\section{Algorithms}

\subsection{Chat Processing Algorithm}
[Describe the algorithm for processing user queries:
\begin{enumerate}
    \item Receive user input (text, images, or documents)
    \item Validate JWT token and authenticate user session
    \item Create or retrieve existing chat thread
    \item Load conversation context and relevant memories
    \item Route to appropriate specialized agent
    \item Execute agent with tool access and external APIs
    \item Stream response tokens via Server-Sent Events
    \item Persist message and update conversation summary
    \item Return complete response to user
\end{enumerate}]

\subsection{Agent Selection Algorithm}
[Describe the router agent's decision process:
\begin{enumerate}
    \item Analyze user query semantics and intent
    \item Check for domain-specific keywords (email, calendar, code, etc.)
    \item Evaluate available agent capabilities
    \item Select most appropriate specialized agent
    \item Fallback to general assistant for ambiguous queries
\end{enumerate}]

\subsection{Redis Queue Processing Algorithm}
[Describe the asynchronous message processing flow:
\begin{enumerate}
    \item Producer publishes message job to Redis Stream
    \item Consumer group claims pending messages
    \item Worker processes message through AI agent pipeline
    \item Status updates published via Redis pub/sub
    \item Completed response stored in database
    \item Failed messages moved to dead letter queue after retries
    \item Client receives updates through SSE subscription
\end{enumerate}]

\subsection{Voice Processing Algorithm}
[Describe the voice interaction pipeline:
\begin{enumerate}
    \item Capture audio stream from client microphone
    \item Buffer audio chunks for processing
    \item Send audio to Sarvam STT for transcription
    \item Process transcribed text through AI agent
    \item Convert AI response to speech using Sarvam TTS
    \item Stream synthesized audio back to client
    \item Handle interruptions and turn-taking
\end{enumerate}]

\subsection{Image Analysis Algorithm}
[Describe the vision processing workflow:
\begin{enumerate}
    \item Receive image upload with optional prompt
    \item Validate file type and size constraints
    \item Encode image to base64 for API transmission
    \item Send to Gemini Vision model with context
    \item Parse structured response (description, extracted text, objects)
    \item Store analysis results with chat message
    \item Return formatted response to user
\end{enumerate}]

\subsection{Memory Management Algorithm}
[Describe context and memory handling:
\begin{enumerate}
    \item Retrieve recent message history from database
    \item Query Supermemory for relevant long-term memories
    \item Combine into structured context for AI agent
    \item After response, extract key facts for memory storage
    \item Periodically summarize old conversations
    \item Prune and consolidate duplicate memories
\end{enumerate}]

\subsection{Google Services Connection Flow}
[Describe the OAuth 2.0 flow for connecting Google apps:
\begin{enumerate}
    \item User initiates connection for a Google service (Gmail, Calendar, etc.)
    \item Redirect to Google consent screen with required scopes
    \item User grants permissions on Google's authorization page
    \item Google redirects back with authorization code
    \item Exchange code for access and refresh tokens
    \item Store encrypted tokens for future API calls
    \item Automatically refresh tokens when expired
\end{enumerate}]

% ==================== CHAPTER 4: IMPLEMENTATION DETAILS ====================
\chapter{Implementation Details}

\section{Backend Implementation}

\subsection{Project Setup}
[Describe the backend setup using Bun, TypeScript, and Express. Include package installation, configuration, and project structure.]

\subsection{Database Design}
[Describe the Prisma schema, database tables, relationships, and migrations. Include the schema for:
\begin{itemize}
    \item User table
    \item Chat table
    \item Message table
    \item Agent table
    \item Document table
    \item Memory table
\end{itemize}]

\subsection{API Endpoints}
[Document all API endpoints:
\begin{itemize}
    \item Authentication routes (/auth)
    \item Chat routes (/chat)
    \item Agent routes (/agents)
    \item Document routes (/documents)
    \item Upload routes (/upload)
    \item App integration routes (/apps)
\end{itemize}]

\subsection{AI Integration}
[Describe the implementation of:
\begin{itemize}
    \item Vercel AI SDK configuration for streaming responses
    \item Google Gemini and OpenAI model integration
    \item Custom tool definitions using Zod schemas
    \item System prompt engineering for each agent type
    \item Hybrid memory using chat history and Supermemory
    \item Multi-agent router pattern implementation
\end{itemize}]

\subsection{Voice Processing Implementation}
[Describe the implementation of:
\begin{itemize}
    \item Sarvam AI client for Indic language STT/TTS
    \item Audio format handling and conversion
    \item LiveKit integration for real-time voice rooms
    \item Token generation for secure voice sessions
    \item Voice-to-agent pipeline coordination
    \item Error handling for audio processing failures
\end{itemize}]

\subsection{Vision and Image Processing}
[Describe the implementation of:
\begin{itemize}
    \item Multer configuration for image uploads
    \item Base64 encoding for API transmission
    \item Gemini Vision API integration
    \item OCR text extraction from documents
    \item Scene description and analysis
    \item Image attachment storage and retrieval
\end{itemize}]

\subsection{Redis Queue Implementation}
[Describe the implementation of:
\begin{itemize}
    \item Redis client connection management
    \item Stream producer for message publishing
    \item Consumer group and worker setup
    \item Job status tracking via pub/sub
    \item Dead letter queue for failed messages
    \item Retry logic with exponential backoff
    \item Health monitoring and metrics collection
\end{itemize}]

\subsection{Google Services Integration}
[Describe the implementation of:
\begin{itemize}
    \item OAuth 2.0 flow for connecting Google accounts
    \item Gmail API for reading, searching, and composing emails
    \item Google Calendar for event creation and schedule queries
    \item Google Docs for document creation and editing
    \item Google Drive for file listing and management
    \item Google Sheets for spreadsheet operations
    \item Token refresh mechanism for persistent access
\end{itemize}]

\subsection{Other External Services}
[Describe the implementation of:
\begin{itemize}
    \item GitHub API for repository operations and code search
    \item Exa for intelligent web search capabilities
    \item Supermemory for long-term memory storage and retrieval
\end{itemize}]

\subsection{Docker Containerization}
[Describe the implementation of:
\begin{itemize}
    \item Multi-stage Dockerfile for backend API
    \item Separate worker Dockerfile for queue processing
    \item Frontend Dockerfile with NGINX serving
    \item Docker Compose for service orchestration
    \item Environment variable management
    \item Volume configuration for persistent data
    \item Health check definitions
    \item Development vs production configurations
\end{itemize}]

\section{Frontend Implementation}

\subsection{Project Setup}
[Describe the frontend setup using React, Vite, and TypeScript. Include component structure and routing.]

\subsection{State Management}
[Describe Zustand store implementation for managing application state]

\subsection{UI Components}
[Describe the implementation of key UI components:
\begin{itemize}
    \item Chat interface
    \item Message bubbles
    \item File upload component
    \item Authentication forms
    \item Navigation menu
    \item Settings panel
\end{itemize}]

\subsection{API Integration}
[Describe how the frontend communicates with the backend API, including error handling and loading states]

\section{Security Implementation}
[Describe security measures:
\begin{itemize}
    \item JWT-based authentication
    \item Data encryption
    \item CORS configuration
    \item Input validation and sanitization
    \item Secure API key management
\end{itemize}]

\section{Code Snippets}
[Include relevant code snippets for key implementations]

% ==================== CHAPTER 5: RESULTS ====================
\chapter{Results}

\section{Screenshots}

[Include screenshots of:
\begin{itemize}
    \item Login and Registration pages
    \item Main chat interface with message history
    \item AI agent responses with tool usage
    \item Voice interaction interface
    \item Image upload and analysis results
    \item Document processing and OCR output
    \item Google service integrations (Gmail, Calendar)
    \item App connection settings
    \item Memory and context display
    \item Mobile responsive views
    \item Docker container status
\end{itemize}

Example references:]

\begin{figure}[h]
\centering
% \includegraphics[width=\textwidth]{screenshot_chat}
\caption{Main Chat Interface}
\label{fig:chat}
\end{figure}

\begin{figure}[h]
\centering
% \includegraphics[width=\textwidth]{screenshot_voice}
\caption{Voice Interaction Interface}
\label{fig:voice_ui}
\end{figure}

\begin{figure}[h]
\centering
% \includegraphics[width=\textwidth]{screenshot_vision}
\caption{Image Analysis Results}
\label{fig:vision_ui}
\end{figure}

\begin{figure}[h]
\centering
% \includegraphics[width=\textwidth]{screenshot_gmail}
\caption{Gmail Integration}
\label{fig:gmail}
\end{figure}

\begin{figure}[h]
\centering
% \includegraphics[width=\textwidth]{screenshot_docker}
\caption{Docker Container Dashboard}
\label{fig:docker_ui}
\end{figure}

\section{Analysis}

\subsection{Performance Metrics}
[Present performance analysis with tables and graphs:
\begin{itemize}
    \item AI response generation time
    \item Voice transcription and synthesis latency
    \item Image processing duration
    \item Redis queue throughput
    \item Database query performance
    \item Docker container resource usage
    \item Concurrent user handling capacity
\end{itemize}]

\begin{table}[h]
\centering
\caption{Text Chat Performance Metrics}
\label{tab:performance}
\begin{tabular}{|l|c|c|c|}
\hline
\textbf{Metric} & \textbf{Minimum} & \textbf{Average} & \textbf{Maximum} \\ \hline
AI Response Time (ms) & & & \\ \hline
First Token Latency (ms) & & & \\ \hline
Database Query (ms) & & & \\ \hline
Memory Usage (MB) & & & \\ \hline
\end{tabular}
\end{table}

\begin{table}[h]
\centering
\caption{Voice Processing Metrics}
\label{tab:voice_perf}
\begin{tabular}{|l|c|c|c|}
\hline
\textbf{Metric} & \textbf{Minimum} & \textbf{Average} & \textbf{Maximum} \\ \hline
STT Latency (ms) & & & \\ \hline
TTS Latency (ms) & & & \\ \hline
End-to-End Voice (ms) & & & \\ \hline
Audio Quality (MOS) & & & \\ \hline
\end{tabular}
\end{table}

\begin{table}[h]
\centering
\caption{Vision Processing Metrics}
\label{tab:vision_perf}
\begin{tabular}{|l|c|c|c|}
\hline
\textbf{Metric} & \textbf{Minimum} & \textbf{Average} & \textbf{Maximum} \\ \hline
Image Analysis (ms) & & & \\ \hline
OCR Extraction (ms) & & & \\ \hline
Scene Description (ms) & & & \\ \hline
\end{tabular}
\end{table}

\begin{table}[h]
\centering
\caption{Docker Resource Usage}
\label{tab:docker_perf}
\begin{tabular}{|l|c|c|c|}
\hline
\textbf{Container} & \textbf{CPU (\%)} & \textbf{Memory (MB)} & \textbf{Image Size (MB)} \\ \hline
Backend API & & & \\ \hline
Worker & & & \\ \hline
Frontend (NGINX) & & & \\ \hline
PostgreSQL & & & \\ \hline
Redis & & & \\ \hline
\end{tabular}
\end{table}

\subsection{Comparison with Existing Systems}
[Compare Kuma with existing AI assistants across key capabilities]

\begin{table}[h]
\centering
\caption{Comparison with Existing AI Assistants}
\label{tab:comparison}
\begin{tabular}{|l|c|c|c|c|c|}
\hline
\textbf{Feature} & \textbf{Kuma} & \textbf{ChatGPT} & \textbf{Google Assistant} & \textbf{Siri} & \textbf{Alexa} \\ \hline
Custom AI Agents & Yes & Limited & No & No & No \\ \hline
Multi-Agent Routing & Yes & No & No & No & No \\ \hline
Gmail Integration & Yes & No & Yes & No & No \\ \hline
Calendar Integration & Yes & No & Yes & Yes & Yes \\ \hline
Document Analysis & Yes & Yes & Limited & Limited & No \\ \hline
Voice Interaction & Yes & Yes & Yes & Yes & Yes \\ \hline
Image Understanding & Yes & Yes & Yes & Yes & Limited \\ \hline
Indic Language Voice & Yes & Limited & Yes & Limited & Limited \\ \hline
Self-Hosted & Yes & No & No & No & No \\ \hline
Open Source & Yes & No & No & No & No \\ \hline
Docker Deployment & Yes & N/A & N/A & N/A & N/A \\ \hline
Long-term Memory & Yes & Yes & Limited & Limited & Limited \\ \hline
\end{tabular}
\end{table}

\subsection{Testing Results}
[Present results from:
\begin{itemize}
    \item Unit testing
    \item Integration testing
    \item User acceptance testing
    \item Performance testing
\end{itemize}]

\subsection{User Feedback}
[If applicable, include user feedback and satisfaction metrics]

% ==================== CHAPTER 6: CONCLUSION ====================
\chapter{Conclusion \& Future Enhancement}

\section{Conclusion}
[Summarize the project achievements:
\begin{itemize}
    \item Successfully developed a comprehensive AI-powered personal assistant platform
    \item Implemented multi-agent architecture with router pattern for intelligent task delegation
    \item Integrated voice interaction with Indic language support using Sarvam AI
    \item Built vision capabilities for image analysis and document understanding
    \item Designed scalable architecture with Redis message queues for reliable processing
    \item Containerized entire stack using Docker for consistent deployment
    \item Integrated multiple productivity services (Gmail, Calendar, Docs, Drive, GitHub)
    \item Created responsive user interface with real-time streaming responses
    \item Achieved production-ready deployment with health monitoring and error recovery
\end{itemize}

Highlight the learning outcomes: practical experience with modern AI frameworks, full-stack development, containerization, and building production-grade applications.]

\section{Future Enhancement}
[Discuss potential future improvements:
\begin{itemize}
    \item Additional service integrations (Slack, Microsoft Office 365, Notion)
    \item Mobile application using React Native for iOS and Android
    \item Fine-tuned models trained on user interaction patterns
    \item Multi-user collaborative workspaces with shared agents
    \item Plugin architecture for community-developed agents and tools
    \item WebSocket-based real-time collaboration features
    \item Kubernetes deployment for enterprise-scale orchestration
    \item On-device voice processing for reduced latency
    \item Offline mode with local LLM support (Ollama integration)
    \item Advanced analytics dashboard for usage insights
    \item Biometric and multi-factor authentication options
    \item IoT device integration for smart home control
    \item Video call integration with screen sharing analysis
    \item Browser extension for contextual assistance
\end{itemize}]

% ==================== BIBLIOGRAPHY ====================
\addcontentsline{toc}{chapter}{Bibliography}

\begin{thebibliography}{99}

\bibitem{langchain}
LangChain Documentation,
\textit{"LangChain: Building applications with LLMs through composability"},
\texttt{https://langchain.com/docs}, 2024.

\bibitem{gemini}
Google DeepMind,
\textit{"Gemini: A Family of Highly Capable Multimodal Models"},
arXiv preprint arXiv:2312.11805, December 2023.

\bibitem{react}
Meta Open Source,
\textit{"React: A JavaScript library for building user interfaces"},
\texttt{https://react.dev}, 2024.

\bibitem{typescript}
Microsoft Corporation,
\textit{"TypeScript: JavaScript with syntax for types"},
\texttt{https://www.typescriptlang.org}, 2024.

\bibitem{prisma}
Prisma Data, Inc.,
\textit{"Prisma: Next-generation Node.js and TypeScript ORM"},
\texttt{https://www.prisma.io}, 2024.

\bibitem{oauth}
D. Hardt,
\textit{"The OAuth 2.0 Authorization Framework"},
RFC 6749, IETF, October 2012.

\bibitem{redis}
Redis Ltd.,
\textit{"Redis Streams: Introduction to Redis Streams"},
\texttt{https://redis.io/docs/data-types/streams/}, 2024.

\bibitem{docker}
Docker Inc.,
\textit{"Docker Documentation: Build, Share, and Run Container Applications"},
\texttt{https://docs.docker.com}, 2024.

\bibitem{vercelai}
Vercel Inc.,
\textit{"AI SDK: The TypeScript toolkit for building AI applications"},
\texttt{https://sdk.vercel.ai/docs}, 2024.

\bibitem{livekit}
LiveKit Inc.,
\textit{"LiveKit: Open source WebRTC infrastructure"},
\texttt{https://livekit.io/docs}, 2024.

\bibitem{sarvam}
Sarvam AI,
\textit{"Sarvam APIs: Speech-to-Text and Text-to-Speech for Indic Languages"},
\texttt{https://www.sarvam.ai}, 2024.

\bibitem{supermemory}
Supermemory,
\textit{"Supermemory: Long-term memory for AI applications"},
\texttt{https://supermemory.ai}, 2024.

\bibitem{bun}
Oven Sh.,
\textit{"Bun: A fast all-in-one JavaScript runtime"},
\texttt{https://bun.sh}, 2024.

\bibitem{agents}
S. Yao et al.,
\textit{"ReAct: Synergizing Reasoning and Acting in Language Models"},
ICLR 2023, arXiv:2210.03629.

\bibitem{transformers}
A. Vaswani et al.,
\textit{"Attention Is All You Need"},
NeurIPS 2017.

\end{thebibliography}

% ==================== APPENDICES ====================
\begin{appendices}
\chapter{Sustainable Development Goals addressed}
{\begin{table}[h]
\centering
\begin{tabular}{|c|c|c|}
\hline 
{\#}&{\bf SDG} & {\bf Level} \\ \hline
1&No Poverty &  \\ \hline
2&Zero Hunger & \\ \hline
3&Good Health and Well-being& \\\hline
4&Quality education &  \\ \hline
5&Gender Quality & \\ \hline
6&Clean water and Sanitation & \\\hline
7&Affordable and Clean Energy & \\ \hline
8&Decent work and Economic Growth & \\ \hline
9&Industry, Innovation and Infrastructure& \\ \hline
10&Reduced Inequalities& \\ \hline
11&Sustainable cities and Communities& \\ \hline
12&Responsible Consumption and production& \\ \hline
13&Climate action& \\ \hline
14&Life below water& \\ \hline
15&Life on Land & \\ \hline
16&Peace, Justice and Strong Institutions& \\ \hline
17&Partnership's for the Goals &\\  \hline
\end{tabular}
%    \caption{Caption}
%    \label{tab:my_label}
\end{table}}
 \\ {\bf Levels: Poor:1, Good :2, Excellent:3}
 
\chapter{Self-Assesment of the Project}
{\begin{table}[h]
%\centering
\begin{tabular}{|c|c|c|c|}
\hline 
{\#}&{\bf PO and PSO} & {\bf Contribution from the Project}&{\bf Level} \\ \hline
1&Engineering Knowledge:
 & & \\ \hline
2&Problem Analysis:&&\\ \hline
3&Design/development of solutions &&\\ \hline
4&Conduct investigations of complex problems: &&\\ \hline
5&
Modern tool usage: &&\\ \hline
6&The Engineer and the world:
&&\\ \hline
7&
Ethics: &&\\ \hline
8&
Individual and Team Work: &&\\ \hline
9&
Communication:&&\\ \hline
10&
Project Management and Finance:&&\\ \hline
11&
Life-long Learning: &&\\ \hline
1&
PSO1 &&\\ \hline
2&
PSO2&&\\ \hline
3&PSO3  &&\\ \hline
\end{tabular}
%    \caption{Caption}
%    \label{tab:my_label}
\end{table}}
\textbf{PSO1: Computer based systems development:} Ability to apply the basic knowledge of database systems, computing, operating system, digital circuits, microcontroller, computer organization and architecture in the design of computer based systems.\\
\textbf{PSO2: Software development:} Ability to specify, design and develop projects, application softwares and system softwares by using the knowledge of data structures, analysis and design of algorithm, programming languages, software engineering practices and open source tools.\\
\textbf{PSO3: Computer communications \\ and Internet applications:} Ability to design and develop network protocols and internet applications by incorporating the knowledge of computer networks, communication protocol engineering, cryptography and network security,   distributed and cloud computing, data mining, big data analytics, ad hoc networks, storage area networks and wireless sensor networks. 
 \\ {\bf Levels: Poor:1, Good :2, Excellent:3}

\chapter{Data Sheet of component 1 }

{\large \textbf{Note: Only include relevant details of the components that are referred w.r.t. project.}}

%% As the data sheets are not edited, to update page number on next appendix i.e., Data Sheet of component 2 use this instruction

\chapter{Data Sheet of component 2}
\setcounter{page}{20}

%% in this, {20}, the no. of pages in Data Sheet of component 1 are 5

%% Enable the above command and in place of {30}, replace with a number that is appropriate.
%% For e.g., Page No. on  Appendix A: Data Sheet of component 1 is 25 and it has 7 sheets.
%% Now replace {30} with {25+7+1} or {33}, this +1 is because Appendix B is a new page and needs to be counted.
 
%% Do it for further Appendix, if any
\end{appendices}

\end{document}
